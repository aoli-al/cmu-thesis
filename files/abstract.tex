\begin{abstract}
Cloud applications prioritize resilience, performance, and scalability. However, the common software engineering practices used to implement these features introduce significant challenges for leveraging traditional program analysis techniques, making it difficult to debug and test cloud applications effectively. For example, exception-handling mechanisms often result in root causes and failures being spatially and temporally distant, complicating failure diagnosis. Similarly, the concurrent and distributed nature of cloud applications renders traditional testing methods for sequential programs insufficient. While advancements in code generation (driven by LLMs and modern IDEs) have significantly improved developers’ ability to write code, their ability to debug effectively and test remains limited.

This thesis addresses these challenges by re-evaluating the trade-offs made by classical program analysis techniques and introducing new methodologies tailored to cloud environments. Our key insight in deriving these innovations is recognizing that cloud applications require a fine balance between precision, performance, and generality. 
% \vyas{insight needs to be technical. this is a nice meta observation for a story abt priorities etc. but not the technical insight that led to the breakthrough in your research. but it may also backfire if its perceived your work is sacrificing accuracy/precision?} 
This thesis argues that to develop practical program analysis for cloud applications, we must prioritize scalability and applicability alongside traditional concerns. Specifically, I present exception dependency analysis and non-intrusive concurrency testing. The exception dependency analysis identifies the root cause for exception-related failures using a lightweight hybrid taint analysis, significantly reducing the overhead but maintaining good accuracy. The non-intrusive concurrency testing platform differs from traditional platforms, which replace concurrency primitives with customized implementations. It carefully inserts additional locks to enforce sequential and deterministic execution and guarantees completeness and soundness. This approach maximizes general-purpose applicability and achieves high performance in testing. 

As part of my proposed research, \leo{discuss future work?}

\end{abstract}